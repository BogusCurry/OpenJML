\chapter{Why specify? Why check?}

\chapter{Background of verification, JML, and OpenJML}

\chapter{Organization of this document}

This document addresses three related topics: how to read, write, and use specifications; the Java Modeling Language (JML) in which specifications are written; and the OpenJML tool that provides editing and checking support for Java programs using JML. 

These three topics are best learned in an interleaved fashion. The tutorial section (Part \ref{part:tutorial}) does just this. It introduces the simpler topics of specification, using Java programs with JML as the specification language, and using OpenJML as the tool to aid editing and checking, with motivating examples. A reader new to JML or to using specifications will find this tutorial to be the easiest introduction to the topics of this book. 

However, it is also useful to have a compact description of each of the JML language and the OpenJML tool. These descriptions are found in Parts \ref{part:jml} and \ref{part:openjml}) respectively. After some introduction, a reader may well want to 
take a break from the tutorial to read through and experiment with the details of JML and OpenJML. Once a reader has graduated from the tutorial and is specifying and verifying new examples, the description of JML serves as a summary of the JML language and the description of OpenJML is the user guide and reference manual for the tool. These two parts stand on there own.

Part \ref{part:contributing} contains information for those interested in contributing to the document. 
Contributions in the form of bug reports and experience reports with substantial use cases or experience in teaching are always welcome; this information can be shared directly with the developers or through the jmlspecs mailing list. 
Part \ref{part:contributing}, however, contains information primarily of interest to those developing and extending the OpenJML code itself.

The final part of the document, Part \ref{part:semantics}, describes details of how OpenJML translates the combination of Java and JML. This is not meant to be read through and is only intended for the reader interested in the detailed semantics of JML and the implementation of OpenJML.

FIXME - what about a mailing list

\chapter{Some details}

\section{Disambiguating `annotation'}. Formal specifications for code are often called annotations; in this document we often
use the term `JML annotations' to refer to specifications written in JML. There is also a specific syntactic construct in Java called 
`annotations': the interfaces labeled with `@' symbols that can modify various syntactic elements of Java. Thus the simple term
'annotation' can be ambiguous. The ambiguity is heightened by the fact that JML annotations, such as \texttt{/*@ pure */}, can 
be expressed as Java annotations, \texttt{@Pure}.

In this document, we will
generally disambiguate the term `annotation' as either 'JML annotation' or 'Java annotation'; if used alone, 'annotation' will generally
mean a JML annotation. 

\section{Syntactic conflicts with @}
For historical reasons, specifications are often written as structured programming language comments, with the \texttt{@} symbol denoting a comment containing specifications. Java comments begin with either \texttt{//} or \texttt{/*}; those comments that
contain JML specifications begin with \texttt{//@} or \texttt{/*@}. 
Similarly, \texttt{//} or \texttt{/*} are also used for comments in C and C++; the ANSI-C Specification Language also uses
\texttt{//@} or \texttt{/*@} to indicate comments containing specifications.

Unfortunately, since the \at symbol is also used for Java annotations, the following problem can arise. Some Java code is written something like (the particular Java annotation and its content are irrelevant)
\boxedexample{
@SuppressWarning("...") \\
class X
}
and then the user comments out the Java annotation without any whitespace:
\boxedexample{
//@SuppressWarning("...") \\
class X
}
Now JML tools will interpret the \texttt{//@} as the beginning of a JML annotation that will generally have parsing errors.

If the user includes whitespace, as in
\boxedexample{
// @SuppressWarning("...")
class X
}
there is then no problem. The workaround for this conflict is to edit the original Java source to include the whitespace. In some situations, placing all JML annotations in a .jml file may solve the problem; however, some tools, including OpenJML, may still parse
the .java file, including the erroneous apparently-JML annotations, even though those annotations are ignored when a .jml file is present.

\section{.jml files and .java files}

-TBD - .jml files hide annotations in .java files, except those in body of methods

\chapter{Quick start to OpenJML}

The details of installing and running OpenJML are presented in Part \ref{part:openjml}. However, an installation of the tool is needed to work through the tutorial. Some impatient readers may also wish to have a quick installation of the tool prior to diving into the full description. This section provides initial installation and use instructions.

\section{Installing OpenJML}

OpenJML is available as a command-line tool and as an Eclipse plug-in. Complete the following steps to install the command-line tool:
\begin{itemize}
\item Create or identify a directory (folder) in which to place the installation. Let \$OPENJML represent the path to this installation directory.
\item Download into \$OPENJML either the zip file at \url{http://jmlspecs.sourceforge.net/openjml.zip} or a gzipped tar file from \url{http://jmlspecs.sourceforge.net/openjml.tar.gz}
\item cd into the directory and either unzip (\texttt{unzip openjml.zip}) or untar (\texttt{tar xvzf openjml.tar.gz}) the downloaded file.
\item
\end{itemize}

\chapter{Other resources}

There are several other useful resources related to JML and OpenJML:
\begin{itemize}
\item \url{http://www.jmlspecs.org} is a web site describing current on JML, including references to many publications, other tools, and links to various groups using JML.
\item \url{http://www.jmlspecs.org/OldReleases/jmlrefman.pdf} is the official reference manual for JML, though it sometimes lags behind agreed-upon changes that are implemented in tools. (FIXME - make a better link)
\item \url{http://www.openjml.org} contains a set of on-line resources for OpenJML
\item \url{http://jmlspecs.sourceforge.net/OpenJMLUserGuide.pdf} is the most current version of this document
\item \url{http://jmlspecs.sourceforge.net/OpenJMLUserGuide.html} is an HTML version, with frames, of this document; \url{http://jmlspecs.sourceforge.net/OpenJMLUserGuide-onepage.html} is the same material in one large HTML page.
\item The source code for OpenJML, the original JML tools, and some other JML projects is contained in the jmlspecs sourceforge project at \url{http://sourceforge.net/projects/jmlspecs}.
\end{itemize}

There are also other tools that make use of JML. An incomplete list follows:
\begin{itemize}
\item Key - FIXME - need url
\item The previous generation of JML tools prior to OpenJML is available at \url{http://www.jmlspecs.org/download.shtml}.
\item FIXME - need others
\end{itemize}