
\chapter{The command-line tool}
\label{CommandLineTool}
\section{Installation and System Requirements}

The command-line tool is supplied as a .zip or a .tar.gz file, downloadable from \url{http://jmlspecs.sourceforge.net/}.
Download the file to a directory of your choice and unzip or untar it in place.
It contains the following files:
\begin{itemize}\nospace
\item openjml.jar - the main jar file for the application
\item jmlruntime.jar - a library needed on the classpath when running OpenJML's runtime-assertion-checking
\item jmlspecs.jar - a library containing specification files
\item openjml-template.properties - a sample file, which should be copied and renamed {\tt openjml.properties}, containing definitions of properties whose values depend on your local system
\item LICENSE.rtf - a copy of the modified GPL license that applies to OpenJDK and OpenJML
\item OpenJMLUserGuide.pdf - this document
\end{itemize}

You must run OpenJML in a Java 1.7 JRE; it is not compatible with Java 1.8 or 1.9.

You should ensure that the {\tt jmlruntime.jar} and {\tt jmlspecs.jar} files remain in the same folder as the {\tt openjml.jar} file.

In addition to OpenJML itself, you will also need a SMT solver if you intend to use the static checking capability of OpenJML.

TBD - more on installing SMT solvers

\section{Running OpenJML}
\label{Running}

To run OpenJML, be sure that the \texttt{java} command uses a 1.7 JVM and use the following command line. Here \textit{\$OPENJML} designates the folder in which the {\tt openjml.jar} file resides.
\boxedexampleZ{
java -jar \textit{\$OPENJML}/openjml.jar {\it <options>} {\it <files>}
}

Here {\it <files>} and {\it <options>} stand for text described below.

The following command is currently a viable alternative as well.
\boxedexampleZ{
java -cp \textit{\$OPENJML}/openjml.jar org.jmlspecs.openjml.Main {\it <options>} {\it <files>}
}

The valid options are listed in Table \ref{Tab:Options} and are described in subsections below. Options and files can appear in any order.

\subsection{Exit values}

When OpenJML runs as a command-line tool, it emits one of several values on exit:
\begin{itemize}[noitemsep,nolistsep]
\item 0 (\texttt{EXIT\_OK}) : successful operation, no errors, there may be warnings (including static checking warnings)
\item 1 (\texttt{EXIT\_ERROR}) : normal operation, but with errors (parsing or type-checking)
\item 2 (\texttt{EXIT\_CMDERR}) : an error in the formulation of the command-line, such as invalid options
\item 3 (\texttt{EXIT\_SYSERR}) : a system error, such as out of memory
\item 4 (\texttt{EXIT\_ABNORMAL}) : a fatal error, such as a program crash or internal inconsistency, caused by an internal bug
\end{itemize}

In the itemized list above, the symbolic names are defined in {\tt org.jmlspecs.openjml.Main}. For example, when executing OpenJML programmatically (cf. section TBD), the user's Java program should use the 
symbol {\tt org.jmlspecsopenjml.Main.EXIT\_ERROR}, rather than a specific integer.

Compiler warnings and static checking warnings will be reported as errors if the \texttt{-Werror} option is used. This may change an \texttt{EXIT\_OK} result to an \texttt{EXIT\_ERROR} result.



\subsection{Files}

In the command templates above, {\it <files>} refers to a list of {\tt .java} or {\tt .jml} files.
Each file must be specified with an absolute file system path or with a path relative
to the current working directory (in particular, not with respect to the classpath or
the sourcepath). 

You can also specify directories on the command line using the {\tt -dir} and {\tt -dirs} options.
The {\tt -dir {\it <directory>}} option indicates that the {\it <directory>} value (an absolute or
relative path to a folder) should be understood as a folder; all {\tt .java} or specification files
recursively (TBD - check recursively) within the folder are included as if they were individually listed on the command-line. The
{\tt -dirs} option indicates that each one of the remaining command-line arguments is interpreted as 
either a source file (if it is a file with a {\tt .java} or {\tt .jml}suffix) or as a folder (if it is a folder)
whose contents are processed as if listed on the command-line. Note that the {\tt -dirs} option must be
the last option.

As described later in section \ref{SpecFiles}, JML specifications for Java programs can be placed either in the 
{\tt .java} files themselves or in auxiliary {\tt .jml} files. The format
of {\tt .jml} files is defined by JML. OpenJML can type-check 
{\tt .jml} files as well as {\tt .java} files if they are placed on the 
command-line. Doing so can be useful to check the syntax in a specific
{\tt .jml} file, but is usually not necessary: when a {\tt .java} file is
processed by OpenJML, the corresponding {\tt .jml} file is automatically found (cf. \ref{TBD}).

\begin{table} \small
%%\centering
\parbox{.5\textwidth}{
\begin{tabular}{|l|p{1.4in}|}
\hline
\multicolumn{2}{|c|}{Options specific to JML} \\
\hline
-- & no more options \\ \hline
-check & [\ref{OptionsTools}] typecheck only ({\tt -command check})\\ \hline
-checkSpecsPath & [\ref{OptionsJML}] warn about non-existent specs path entries\\ \hline
-command {\it <action>}& [\ref{OptionsTools}] which action to do: check esc rac compile\\ \hline
-compile & [\ref{OptionsTools}] TBD\\ \hline
-counterexample & [\ref{OptionsESC}] show a counterexample for failed static checks\\ \hline
-dir {\it <dir>} & [\ref{OptionsJML}] argument is a folder or file \\ \hline
-dirs & [\ref{OptionsJML}] remaining arguments are folders or files\\ \hline
-esc & [\ref{OptionsTools}] do static checking ({\tt -command esc}) \\ \hline
-internalRuntime & [\ref{OptionsJML}] add internal runtime library to classpath \\ \hline
-internalSpecs & [\ref{OptionsJML}] add internal specs library to specspath \\ \hline
-java & [\ref{OptionsTools}] use the native OpenJDK tool\\ \hline
-jml & [\ref{OptionsTools}] process JML constructs \\ \hline
-jmldebug & [\ref{OptionsDebugging}] very verbose output (includes -progress) \\ \hline
-jmlverbose & [\ref{OptionsDebugging}] JML-specific verbose output\\ \hline
-keys & [\ref{OptionsJML}] define keys for optional annotations \\ \hline
-method & \\ \hline
-nonnullByDefault & [\ref{OptionsJML}] values are not null by default \\ \hline
-normal & [\ref{OptionsDebugging}] \\ \hline
-nullableByDefault & [\ref{OptionsJML}] values may be null by default\\ \hline
-progress & [\ref{OptionsDebugging}] \\ \hline
-purityCheck & [\ref{OptionsJML}] check for purity \\ \hline
-quiet & [\ref{OptionsDebugging}] no informational output \\ \hline
-rac & [\ref{OptionsTools}] compile runtime assertion checks ({\tt -command rac})\\ \hline
-racCheckAssumptions & [\ref{OptionsRAC}] enables (default on) checking assume statements as if they were asserts \\ \hline
-racCompileToJavaAssert & [\ref{OptionsRAC}] compile RAC checks using Java asserts \\ \hline
-racJavaChecks & [\ref{OptionsRAC}] enables (default on) performing JML checking of violated Java features \\ \hline
\end{tabular}
}
\qquad
%%\centering
\parbox{.5\textwidth}{

\begin{tabular}{|l|p{1.4in}|}
\hline
\multicolumn{2}{|c|}{JML options, continued} \\
\hline
-racShowSource & [\ref{OptionsRAC}] includes source location in RAC warning messages \\ \hline
-showNotImplemented & warn if feature not implemented\\ \hline
-specspath & [\ref{OptionsJML}] location of specs files\\ \hline
-stopIfParseErrors & stop if there are any parse errors \\ \hline
-subexpressions & [\ref{OptionsESC}] show subexpression detail for failed static checks\\ \hline
-trace & [\ref{OptionsESC}] show a trace for failed static checks\\ \hline

\end{tabular}

\vspace*{.5in}

\begin{tabular}{|l|p{1.4in}|}
\hline
\multicolumn{2}{|c|}{Options inherited from Java} \\
\hline
-Akey & \\ \hline
-bootclasspath {\it <path>}& See Java documentation. \\ \hline
-classpath {\it <path>}& location of class files \\ \hline
-cp {\it <path>}& location of class files\\ \hline
-d {\it <directory>} & location of output class files\\ \hline
-encoding {\it <encoding>} & \\ \hline
-endorsedirs {\it <dirs>} & \\ \hline
-extdirs {\it <dirs>} & \\ \hline
-deprecation & \\ \hline
-g & \\ \hline
-help & output help information\\ \hline
-implicit & \\ \hline
-J{\it <flag>} & \\ \hline
-nowarn & show only errors, no warnings \\ \hline
-proc & \\ \hline
-processor {\it <classes>} & \\ \hline
-processorpath {\it <path>} & where to find annotation processors\\ \hline
-s {\it <directory>} & location of output source files\\ \hline
-source {\it <release>} & the Java version of source files\\ \hline
-sourcepath {\it <path>} & location of source files\\ \hline
-target {\it <release>} & the Java version of the output class files\\ \hline
-X & Java non-standard extensions\\ \hline
-verbose & verbose output \\ \hline
-version & output (OpenJML) version\\ \hline
-Werror & treat warnings as errors \\ \hline

\end{tabular}
}
\caption{OpenJML options. See the text for more detail on each option.}
\label{Tab:Options}
\end{table}

\subsection{Specification files}
\label{SpecFiles}

JML specifications for Java classes (either source or binary) are written in files with a {\tt .jml} suffix or are written directly in the source {\tt .java} file.
When OpenJML needs specifications for a given class, it looks for a {\tt .jml} file on the specspath. If one is not found, OpenJML then looks for a {\tt .java}
file on the specspath. Note that this rule requires that source files (that have specifications you want to use) must be listed on the specspath. Note also that there 
need not be a source file; a {\tt .jml} file can be (and often is) used to provide specifications for class files.

Previous versions of JML had a more complicated scheme for constructing specifications for a class involving refinements, multiple specification files, and various prefixes. This complicated process is now deprecated and no longer supported.

[ TBD: some systems might find the first .java or .jml file on the specspath and use it, even if there were a .jml file later.]


\subsection{Annotations and the runtime library}

JML uses Java annotations as introduced in Java 1.6. Those annotation classes are in the package
{\tt org.jmlspecs.annotation}. In order for files using these annotations to be processed by Java,
the annotation classes must be on the classpath. They may also be required when a compiled Java program
that uses such annotations is executed. In addition, running a program that has JML runtime assertion
checks compiled in will require the presence of runtime classes that define utility functions used by the assertion checking code.

Both the annotation classes and the runtime checking classes are provided in a library named {\tt jmlruntime.jar}.  The distribution of OpenJML contains this library, as well as containing a
version of the library within {\tt openjml.jar}. When OpenJML is applied to a set of classes, by default it finds a version of the runtime classes and appends the location of the runtime classes
to the classpath.

You can prevent OpenJML from automatically adding {\tt jmlruntime.jar} to the classpath with the
option {\tt -noInternalRuntime}. If you use this option, then you will have to supply your own
annotation classes and (if using Runtime Assertion Checking) runtime utility classes on the classpath. You may wish to do this, for example, if you have newer versions of the annotation
classes that you are experimenting with. You could simply put them on the classpath, since they
would be in front of the automatically added classes and used in favor of default versions;
however, if you want to be sure that the default versions are not present, use the {\tt -noInternalRuntime} option.

The symptom that no runtime classes are being found at all is error messages that complain that
the {\tt org.jmlspecs.annotation} package is not found.


\subsection{Java properties and the {\tt openjml.properties} file}

OpenJML uses a number of properties that may be defined in the environment;
these properties are typically characteristics of the local environment that vary among different users or different installations. 
They can also be to set default values of options, so they do not need to be set on the command-line. An example is the file system location of a particular solver.

The tool looks for a file named {\tt openjml.properties} in several locations. It loads the
properties it finds in each of these, in order, so later definitions will supplant earlier ones.
\begin{itemize}\nospace
\item System properties, including those defined with {\tt -D} options on the command-line
\item On the system classpath
\item In the users home directory (the value of the Java property {\tt user.home}
\item In the current working directory (the value of the Java property {\tt user.dir}
\end{itemize}
[TBD: Check the above]

The properties that are currently recognized are these:
\begin{itemize}\nospace
%% TBD \item {\tt openjml.option.{\it <key>}}, where {\it <key>} is the name of a command-line option and the value is the value of the option as if it were specified on the command-line; values actually specified on the command-line override any specified in a properties file
\item {\tt openjml.defaultProver} - the value is the name of the prover to use by default
\item {\tt openjml.prover.{\it <name>}}, where {\it <name>} is the name of a prover, and
the value is the file system path to the executable to be invoked for that prover
\end{itemize}
[TBD: Check the above]

The distribution includes a file named {\tt openjml-template.properties} that contains stubs for all the recognized options.
You should copy that file, rename it as {\tt openjml.properties}, and edit it to reflect your system configuration.
(If you are an OpenJML developer, take care not to commit your system's {\tt openjml.properties} file into the OpenJML shared SVN repository.)

\subsection{OpenJML options}

There are many options that control or modify the behavior of OpenJML. Some of these are inherited from the Java compiler on 
which OpenJML is based. Options for the command-line tool are expressed as standard command-line options. In the Eclipse GUI, the values of options are set on a typical Eclipse preference or properties page for OpenJML.

The command-line options follow the style of the OpenJDK compiler --- they begin 
with a single hyphen and there are no two-hyphen versions.
OpenJML (but not OpenJDK) options that require a parameter may either use an = followed directly by the argument with no whitespace or 
may provide the argument as the subsequent entry of the argument list. 
For example, either \texttt{--racbin=output} or
\texttt{--racbin output} is permitted. If the argument is optional 
but present, the = form must be used. Values of options that contain whitespace must be quoted as appropriate for the operating system being used.

Options that are boolean in nature can be enabled and disabled by either
\begin{itemize}[noitemsep,nolistsep]
\item adding a prefix -no, as in \texttt{-showRacSource} and \texttt{-no-showRacSource}
\item or using the = form, as in \texttt{-showRacSource=true} and \texttt{-showRacSource=false}
\end{itemize}

\subsubsection{Options: Operational modes}
These operational modes are mutually exclusive.
\begin{itemize}[noitemsep,nolistsep]
\item \textbf{-jml} (default) : use the OpenJML implementation to process the listed files, including embedded JML comments and any .jml files
\item \textbf{-no-jml}: uses the OpenJML implementation to type-check and possibly compile the listed files, but ignores all JML annotations in those files
\item \textbf{-java}: processes the command-line options and files using only OpenJDK functionality. No OpenJML functionality is invoked. Must be the first option and overrides the others.
\end{itemize}

\subsubsection{Options: JML tools}
\label{OptionsTools}
The following mutually exclusive options determine which OpenJML tool is applied to the input files. They presume that the \texttt{-jml} mode is in effect.
\begin{itemize}
\item \textbf{-command} {\it <tool>} : initiates the given function; the value of {\it <tool>} may be one of {\tt check}, {\tt esc}, {\tt rac}, TBD.
The default is to use the OpenJML tool to do only typechecking of Java and JML in the source files.
\item \textbf{-check} : causes OpenJML to do only type-checking of the Java and JML in the input files (alias for \texttt{-command=check})
\item \textbf{-compile} : TBD
\item \textbf{-esc} : causes OpenJML to do (type-checking and) static checking of the JML specifications against the implementations in the input files  (alias for \texttt{-command=esc})
\item \textbf{-rac} : compiles the given Java files as OpenJDK would do, but with JML checks included for checking at runtime  (alias for \texttt{-command=rac})
\item \textbf{-doc} : executes javadoc but adds JML specifications into the javadoc output files.  (alias for \texttt{-command=doc})\textit{Not yet implemented.} 
\end{itemize}


\subsubsection{Options: Finding files and classes: class, source, and specs paths}
\label{OptionsPaths}

A common source of confusion is the various different paths used to find files, specifications and classes in OpenJML.
OpenJML is a Java application and thus a {\it classpath} is used to find the classes that constitute the OpenJML application;
but OpenJML is also a tool that processes Java files, so it uses a (different) classpath to find the files that it is processing. 
As is the case for other Java applications, a {\it <path>} contains a sequence of individual paths to folders or jar files, separated
by the path separator character (a semicolon on Windows systems and a colon on Unix and MacOSX systems).
You should distinguish the following:
\begin{itemize}
\item the classpath used to run the application: specified by one of
\begin{itemize}\nospace
\item the {\tt CLASSPATH} environment variable
\item the .jar file given with the {\tt java -jar} form of the command is used
\item the value for the {\tt -classpath} (equivalently, {\tt -cp}) option when OpenJML is run with the
{\tt java -cp openjml.jar org.jmlspecs.openjml.Main} command
\end{itemize}
This classpath is not of much concern to OpenJML, but is the classpath that Java users will be familiar with.
The value is implicitly given in the {\tt -jar} form of the command. The application classpath is explicitly given in the alternate form of the command,
and it may be omitted; if it is omitted, the value of the system property {\tt CLASSPATH} is used and it must contain the {\tt openjml.jar} library.

\item the classpath used by OpenJML. This classpath determines where OpenJML will find .class files for classes 
referenced by the {\tt .java} files it is processing. The classpath is specified by\\
\centerline{\tt -classpath {\it <path>}}
or \\
\centerline{\tt -cp {\it <path>}}
{\it after} the executable is named on the commandline.  That is,
\boxedexampleB{
java -jar openmjml.jar -cp {\it <openjml-classpath>} ...
}
or
\boxedexampleB{
java -cp openjml.jar org.jmlspecs.openjml.Main -cp {\it <openjml-classpath>} ...
}
If the OpenJML classpath is not specified, its value is the same as the application classpath.

\item the OpenJML sourcepath - The sourcepath is used by OpenJML as the list of locations in which to find {\tt .java} 
files that are referenced by the files being processed. For example, if a file on the command-line, say {\tt T.java},
refers to another class, say {\tt class U}, that is not listed on the command-line, then {\tt U} must be found.  OpenJML (just as is done by the Java compiler) will look for a source file for {\tt U} in the sourcepath and a class file for {\tt U} in the classpath.
If both are found then TBD.

The OpenJML sourcepath is specified by the {\tt -sourcepath {\it <path>}} option. If it is not specified, the value for the sourcepath is taken to be the same as the OpenJML classpath.

In fact, the sourcepath is rarely used.  Users often will specify a classpath containing both {\tt .class} and {\tt .java} files; by 
not specifying a sourcepath, the same path is used for both {\tt .java} and {\tt .class} files. This is simpler to write, but does mean
that the application must search through all source and binary directories for any particular source or binary file.

\item the OpenJML specspath - The specspath tells OpenJML where to look for specification ({\tt .jml}) files. It is specified with the {\tt -spacspath {\it <path>}} option. If it is not specified, the value for the specspath is the same as the value for the sourcepath.  In addition, by default, the specspath
has added to it an internal library of specifications.  These are the existing (and incomplete) specifications of the Java standard library classes.

The addition of the Java specifications to the specspath can be disabled by using the {\tt -noInternalSpecs} option.  For example. if you
have your own set of specification files that you want to use instead of the internal library, then you should use the {\tt -noInternalSpecs} option and a {\tt -specspath} option with a path that includes your own specification library.

Note also that often source ({\tt .java}) files contain specifications as well. Thus, if you are specifying a specspath yourself, you should
be sure to include directories containing source files in the specspath; this rule also includes the {\tt .java} files that appear on the 
command-line: they also should appear on the specspath.

TBD - describe what happens if the above guidelines are not followed. (Can we make this more user friendly).

\end{itemize}

\paragraph{The {\tt -noInternalSpecs} option.} As described above, this option turns off the automatic adding of the internal specifications library to the specspath. If you use this option, it is your responsibility to provide an alternate specifications library for the standard
Java class library. If you do not you will likely see a large number of static checking warnings when you use Extended Static Checking to check the implementation code against the specifications.

The internal specifications are written for programs that conform to Java 1.7.  [ TBD - change this to adhere to the {\tt -source} option?] 
[TBD - what about the specs in jmlspecs for different source levels.]

\subsubsection{Options: OpenJML options applicable to all OpenJML tools }
\label{OptionsJML}

\begin{itemize}
\item \textbf{-dir} \textit{<folder>} : abbreviation for listing on the command-line all of the .java files in the given folder, and its subfolders; if the argument is a file, use it as is
\item \textbf{-dirs} : treat all subsequent command-line arguments as if each were the argument to \texttt{-dir}
\item \textbf{-specspath} \textit{<path>} : defines the specifications path, cf. section TBD
\item \textbf{-keys} \textit{<keys>} : the argument is a comma-separated list of options JML keys (cf. section TBD)
\item \textbf{-strictJML} : warns about an OpenJML extensions to standard JML
\end{itemize}

\begin{itemize}
\item \textbf{-nullableByDefault} : sets the global default to be that all declarations are implicitly \texttt{@Nullable}
\item \textbf{-nonnullByDefault} : sets the global default to be that all 
declarations are implicitly \texttt{@NonNull} (the default)
\item \textbf{-purityCheck} : turns on (default is on) purity checking (recommended since the Java library specifications are not complete for \texttt{@Pure} declarations)
\end{itemize}

\subsubsection{Options: Extended Static Checking}
\label{OptionsESC}

These options apply only when performing ESC:
\begin{itemize}
\item \textbf{-prover} \textit{<prover>} : the name of the prover to use: one of z3\_4\_3, cvc4, yices2
\item \textbf{-exec} \textit{<file>} : the path to the executable corresponding to the given prover
\item \textbf{-boogie} : enables using boogie (-prover option ignored; -exec must specify the Z3 executable for Boogie to use)
\item \textbf{-method} \textit{<methodlist>} : a comma-separated list of method names to
check (default is all methods in all listed classes)
\item \textbf{-exclude} \textit{<methodlist>} : a comma-separated list of method names to exclude from checking
\item \textbf{-checkFeasibility} \textit{<where>} : checks feasibility of the program at various points --- a comma-separated list of
one of \texttt{none}, \texttt{all}, \texttt{exit} [TBD, finish list,  give default]
\item \textbf{-escMaxWarnings} \textit{<int>} : the maximum number of assertion violations to look for; the argument is either a positive integer or \texttt{All}; the default is \texttt{All}
\item \textbf{-counterexample} : prints out a counterexample for failed proofs
\item \textbf{-trace} : prints out a counterexample trace for each failed assert (includes -counterexample)
\item \textbf{-subexpressions} : prints out a counterexample trace with model values for each subexpression (includes -trace)
\end{itemize}

\subsection{Options: Runtime Assertion Checking}
\label{OptionsRAC}

These options apply only when doing RAC:
\begin{itemize}
\item \textbf{-showNotExecutable} : warns about the use of features that are not executable (and thus ignored); turn off with \texttt{-no-shownotExecutable}
\item \textbf{-showRacSource} : enables including source code information in RAC error messages (default is enabled; disable with \texttt{-no-showRacSource})
\item \textbf{-racCheckAssumptions} : enables checking \texttt{assume} statements as if they were asserts (default is enabled; disable with\texttt{-no-racCheckAssumptions})
\item \textbf{-racJavaChecks} : enables performing JML checking of violated Java features (which will just proceed to throw an exception anyway) (default is enabled; disable with \texttt{-no-racJavaChecks})
\item \textbf{-racCompileToJavaAssert} : compile RAC checks using Java asserts (which must then be enabled using \texttt{-ea}) (default is disabled; disable with \texttt{-no-racCompileToJavaAssert})
\end{itemize}

\subsubsection{Options: Version of Java language or class files}

\begin{itemize}
\item \textbf{-source} {\it <level>} : this option specifies the Java version of the source files, with values of {\tt 1.4}, {\tt 1.5}, {\tt 1.6}, {\tt 1.7}... or {\tt 4}, {\tt 5}, {\tt 6}, {\tt 7}, ... . This controls whether some syntax features  
(e.g. annotations, extended for-loops, autoboxing, enums) are permitted. The default is the most recent version
of Java, in this case 1.7.  Note that the classpath should include the Java library classes that
correspond to the source version being used.

\item \textbf{-target} {\it <level>} : this option specifies the Java version of the output class files
\end{itemize}


\subsubsection{Options: Other Java compiler options applicable to OpenJML}

All the OpenJDK compiler options apply to OpenJML as well. The most commonly used or important OpenJDK options are listed here.

These options control where output is written:
\begin{itemize}
\item \textbf{-d} {\it <dir>} : specifies the directory in which output class files are placed; the directory must already exist
\item \textbf{-s} {\it <dir>} : specifies the directory in which output source files are placed; such as those produced by annotation processors; the directory must already exist
\end{itemize}

These are Java options relevant to OpenJML whose meaning is unchanged in OpenJML. 
\begin{itemize}[noitemsep,nolistsep]
\item \textbf{-cp} or \textbf{-classpath}: the parameter gives the classpath to use to find unnamed but referenced class files (cf. section TBD)
\item \textbf{-sourcepath}: the parameter gives the sequence of directories in which to find source files for unnamed but referenced classes (cf. section TBD)
\item \textbf{-deprecation}: enables warnings about the use of deprecated features (applies to deprecated JML features as well)
\item \textbf{-nowarn}: shuts off all compiler warnings, \textit{including the static check warnings produced by ESC}
\item \textbf{-Werror}: turns all warnings into errors, including JML (and static check) warnings
\item \textbf{@\textit{filename}}: the given \textit{filename} contains a list of arguments
\item \textbf{-source}: specifies the Java version to use (default 1.7)
\item \textbf{-verbose}: turn on Java verbose output
\item \textbf{-Xprefer:source} or \textbf{-Xprefer:newer}: when both a .java and a .class file are present, 
whether to choose the .java (source) file or the file that has the more recent modification time [ TBD - check that this works ]
\item \textbf{-stopIfParseErrors}: if enabled (disabled by default), processing stops after parsing if there are any parsing errors (TBD - check this, describe the default)
\end{itemize}




Other Java options, whose meaning and use is unchanged from javac:
\begin{itemize}
\item \textbf{@}\textit{<filename>} : reads the contents of \textit{<filename>} as a sequence of command-line arguments (options, arguments and files)
\item \textbf{-Akey}
\item \textbf{-bootclasspath}
\item \textbf{-encoding}
\item \textbf{-endorsedirs}
\item \textbf{-extdirs}
\item \textbf{-g}
\item \textbf{-implicit}
\item \textbf{-J}
\item \textbf{-nowarn} : only print errors, not warnings, \textit{including not printing static check warnings}
\item \textbf{-Werror} : turns all warnings into errors
\item \textbf{-X}... : Java's extended options
\end{itemize}

These Java options are discussed elsewhere in this document:
\begin{itemize}
\item \textbf{-cp} \textit{<path>} or \textbf{-classpath} \textit{<path>} : section \ref{OptionsPaths}
\item \textbf{-sourcepath} \textit{<path>} : section \ref{OptionsPaths}
\item \textbf{-verbose} : section \ref{OptionsDebugging}
\item \textbf{-source} :  
\item \textbf{-target} :  
\end{itemize}


\subsubsection{Options: Information and debugging}
\label{OptionsDebugging}
These options print summary information and immediately exit (despite the presence of other command-line arguments):
\begin{itemize}
\item \textbf{-help} : prints out help information about the command-line options
\item \textbf{-version} : prints out the version of the OpenJML tool software
\end{itemize}
The following options provide different levels of verboseness. If more than one is specified, the last one present overrides earlier ones.
\begin{itemize}
\item \textbf{-quiet} : no informational output, only errors and warnings
\item \textbf{-normal} : (default) some informational output, in addition to errors and warnings
\item \textbf{-progress} : prints out summary information as individual files are processed (includes -normal)
\item \textbf{-verbose} : prints out verbose information about the Java processing in OpenJDK (does not include other OpenJML information)
\item \textbf{-jmlverbose} : prints out verbose information about the JML processing (includes -verbose and -progress)
\item \textbf{-jmldebug} : prints out (voluminous) debugging information (includes -jmlverbose)
\item \textbf{-verboseness} {\it <int>} : sets the verboseness level to a value from 0 - 4, corresponding to -quiet, -normal, -progress, -jmlverbose, -jmldebug
\end{itemize}

Other debugging options:
\begin{itemize}
\item \textbf{-show} : prints out rewritten versions of the Java program files for informational and debugging purposes
\item \textbf{-showNotImplemented} : prints warnings about JML features that are ignored because they are not implemented; 
disable with \texttt{-no-showNotImplemented}; the default is disabled.
\end{itemize}

An option used primarily for testing:
\begin{itemize}[noitemsep,nolistsep]
\item \textbf{-jmltesting}: adjusts the output so that test output is more stable
\end{itemize}

\subsection{Options related to Static Checking}
\begin{itemize}
\item -counterexample
\item -trace
\item -subexpressions
\item -method
\end{itemize}

\subsection{Options related to parsing and typechecking}
\begin{itemize}
\item -Werror
\item -nowarn
\item -stopIfParseError
\item -checkSpecsPath
\item -purityCheck
\item -nonnullbydefault
\item -nullablebydefault
\item -keys
\end{itemize}

\subsection{Java options related to annotation processing}
\begin{itemize}
\item -proc
\item -processor
\item -processorpath
\end{itemize}

\subsection{Other JML Options}
\begin{itemize}
\item -roots
\end{itemize}


\paragraph{General options}
\begin{itemize}[noitemsep,nolistsep]
\item \textbf{-dir}: Indicates that its argument is a directory. All the .java and .jml files in the directory and its subdirectories are processed. (TBD - is this necessary?)
\item \textbf{-dirs}: Indicates that all subsequent command-line arguments are directories, to be processed as for \texttt{-dir}, until an argument is reached that begins with a hyphen. 
\item \textbf{-specspath}: the parameter gives the sequence of directories in which to find .jml specification files for unnamed but referenced classes (cf. section TBD)
\item \textbf{-checkSpecsPath}: if enabled (the default), warns about \texttt{specspath} elements that do not exist
\item \textbf{-keys}: comma-separated list of the optional JML comment keys to enable (empty by default)
\item \textbf{-strictJML}: (disabled by default) warns about the use of any OpenJML extensions to standard JML; disable with -no-strictJML
\item \textbf{-showNotImplemented}: (disabled by default) warns about the use of features that are not implemented; disable with -no-showNotImplemented
\end{itemize}


\textit{This section will be completed later.} %% TBD


