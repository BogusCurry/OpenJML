\chapter{Introduction}

\section{OpenJML}

OpenJML is a tool for processing Java Modeling Language (JML) specifications of Java programs. 
The tool parses and type-checks the specifications and
performs static or run-time checking of the validity of the specifications. 
Other tools are anticipated, such as test case generation and
 an enhanced version of javadoc that includes the specifications in the
javadoc documentation.

The functionality is available 
\begin{itemize}
\item as a command-line tool to do type-checking, static checking or runtime checking,
\item as an Eclipse plug-in to perform those tasks, and
\item programmatically from a user's Java program
\end{itemize}

OpenJML uses program specifications written in JML, the Java Modeling Language, and
it is constructed by extending OpenJDK, the open source Java compiler.
OpenJML currently requires Java 1.7 to run.
Releases of Java can be obtained from the Oracle release site (\url{http://www.oracle.com/technetwork/java/javase/downloads}).

The source code for OpenJML is kept in SourceForge as a module of the 
JML Project (\url{https://sourceforge.net/p/jmlspecs/code/HEAD/tree/}).
The JMLAnnotations and Specs projects (also modules of JML) are used by OpenJML.
Information about creating a developer's environment for the OpenJML source
can be found below (section \ref{Development}).
 
\subsection{Command-line tool}

The OpenJML command line tool can be downloaded from
\url{http://jmlspecs.sourceforge.net/openjml.tar.gz}.

The command line tool is described in chapter \ref{CommandLineTool}.

\subsection{Eclipse plug-in}

The Update site for the Eclipse plug-in that encapsulates the OpenJML tool
is \url{http://jmlspecs.sourceforge.net/openjml-updatesite}.

The plug-in is described in section \ref{EclipsePlugin} and in the online documentation available in Eclipse Help.


\subsection{Development of OpenJML}
\label{Development}

Developers wishing to contribute to OpenJML can retrieve a project-set file to download source code from SVN and create the corresponding projects within Eclipse from 
\url{http://jmlspecs.sourceforge.net/OpenJML-projectSet.psf}.
<p>Alternately, the set of SVN commands needed to checkout all the pieces of the
OpenJML source code into the directory structure expected by Eclipse is found at this link: 
\url{http://jmlspecs.sourceforge.net/svn_commands}.

The general instructions for setting up a development environment are found at the JML wiki: \url{https://sourceforge.net/apps/trac/jmlspecs/wiki/OpenJmlSetup}.



\section{JML}

The Java Modeling Language (JML) is a language that enables logical assertions
to be made about Java programs. The assertions are expressed as structured 
Java comments or Java annotations. Various tools can then read the JML 
information and do static checking, runtime checking, display for documentation,
or other useful tasks.

More information about JML can be found on the JML web site: 
\url{http://www.jmlspecs.org}.
The information includes publications, a list of groups using or contributing to JML, 
mailing lists, etc.
There is also a SourceForge project for JML : \url{https://sourceforge.net/projects/jmlspecs/}.

\section{OpenJDK}

OpenJDK (\url{http://openjdk.net}) is the project that produces the Java JDK and JRE releases.
OpenJML extends OpenJDK to produce the OpenJML tools. OpenJML is a 
fully encapsulated, stand-alone tool, so the OpenJDK foundation is only of interest to OpenJML developers.
Users, however, can be assured that OpenJML is built on 'official' Java tooling and can readily stay up 
to date with changes in the Java language.

\section{License}

The OpenJML command-line tool is built from OpenJDK, which is licensed under GPLv.2 (\url{http://openjdk.java.net/legal/}).
Hence OpenJML is correspondingly licensed.

The OpenJML plug-in is a pure Eclipse plug-in, and therefore is not required to be licensed under the EPL.


The source code for both tools is available as a SourceForge project at
\url{https://sourceforge.net/p/jmlspecs/code/HEAD/tree/OpenJML/trunk}.
