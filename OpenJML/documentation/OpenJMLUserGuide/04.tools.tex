
\chapter{OpenJML tools}

TBD - exit codes

\section{Options controlling OpenJML behavior}
There are many options that control or modify the behavior of OpenJML. Some of these are inherited from the Java compiler on 
which OpenJML is based. Options for the command-line tool are expressed as standard command-line options. In the Eclipse GUI, the values of options are set on a typical Eclipse preference or properties page for OpenJML.

The command-line options follow the style of the OpenJDK compiler --- they begin 
with a single hyphen and there are no two-hyphen versions.
OpenJML (but not OpenJDK) options that require a parameter may either use an = followed directly by the argument with no whitespace or 
may provide the argument as the subsequent entry of the argument list. 
For example, either \texttt{--racbin=output} or
\texttt{--racbin output} is permitted. If the argument is optional 
but present, the = form must be used. Values of options that contain whitespace must be quoted as appropriate for the operating system being used.

Options that are boolean in nature can be enabled and disabled by either
\begin{itemize}[noitemsep,nolistsep]
\item adding a prefix -no, as in \texttt{-showRacSource} and \texttt{-no-showRacSource}
\item or using the = form, as in \texttt{-showRacSource=true} and \texttt{-showRacSource=false}
\end{itemize}

\paragraph{Informational options}
\begin{itemize}[noitemsep,nolistsep]
\item \textbf{-help}: gives information about the command-line options and exits, with no further processing
\item \textbf{-version}: gives the version of this OpenJML tool and exits, with no further processing
\end{itemize}


\paragraph{OpenJML operational modes (mutually exclusive)}
\begin{itemize}[noitemsep,nolistsep]
\item \textbf{-jml} (default) : use the OpenJML implementation to process the listed files, including embedded JML comments and any .jml files
\item \textbf{-no-jml}: uses the OpenJML implementation to type-check and possibly compile the listed files, but ignores all JML annotations in those files
\item \textbf{-java}: processes the command-line options and files using only OpenJDK functionality. No OpenJML functionality is invoked. Must be the first option and overrides the others.
\end{itemize}

\paragraph{OpenJML tools (mutually exclusive) --- presumes \texttt{-jml}}
\begin{itemize}[noitemsep,nolistsep]
\item \textbf{-check}: (default) runs JML parsing and type-checking
\item \textbf{-esc}: runs extended static checking
\item \textbf{-rac}: compiles files with runtime assertions
\item \textbf{-doc}: runs the jmldoc tool (not yet implemented)
\item \textbf{-command \textit{command}}: runs the given command, for arguments
\texttt{check}, \texttt{esc}, \texttt{rac}, or \texttt{doc}; 
the default is \texttt{check}
\end{itemize}

\paragraph{Relevant Java compiler options}
All the OpenJDK compiler options apply to OpenJML as well. The most commonly used or important OpenJDK options are listed here.
\begin{itemize}[noitemsep,nolistsep]
\item \textbf{-cp} or \textbf{-classpath}: the parameter gives the classpath to use to find unnamed but referenced class files (cf. section TBD)
\item \textbf{-sourcepath}: the parameter gives the sequence of directories in which to find source files for unnamed but referenced classes (cf. section TBD)
\item \textbf{-d}: specifies the output directory for compiled files - the directory must exist
\item \textbf{-deprecation}: enables warnings about the use of deprecated features (applies to deprecated JML features as well)
\item \textbf{-nowarn}: shuts off all compiler warnings, \textit{including the static check warnings produced by ESC}
\item \textbf{-Werror}: turns all warnings into errors, including JML (and static check) warnings
\item \textbf{@\textit{filename}}: the given \textit{filename} contains a list of arguments
\item \textbf{-source}: specifies the Java version to use (default 1.7)
\item \textbf{-verbose}: turn on Java verbose output
\item \textbf{-Xprefer:source} or \textbf{-Xprefer:newer}: when both a .java and a .class file are present, 
whether to choose the .java (source) file or the file that has the more recent modification time [ TBD - check that this works ]
\item \textbf{-stopIfParseErrors}: if enabled (disabled by default), processing stops after parsing if there are any parsing errors (TBD - check this, describe the default)
\end{itemize}

\paragraph{General options}
\begin{itemize}[noitemsep,nolistsep]
\item \textbf{-dir}: Indicates that its argument is a directory. All the .java and .jml files in the directory and its subdirectories are processed. (TBD - is this necessary?)
\item \textbf{-dirs}: Indicates that all subsequent command-line arguments are directories, to be processed as for \texttt{-dir}, until an argument is reached that begins with a hyphen. 
\item \textbf{-specspath}: the parameter gives the sequence of directories in which to find .jml specification files for unnamed but referenced classes (cf. section TBD)
\item \textbf{-checkSpecsPath}: if enabled (the default), warns about \texttt{specspath} elements that do not exist
\item \textbf{-keys}: comma-separated list of the optional JML comment keys to enable (empty by default)
\item \textbf{-strictJML}: (disabled by default) warns about the use of any OpenJML extensions to standard JML; disable with -no-strictJML
\item \textbf{-showNotImplemented}: (disabled by default) warns about the use of features that are not implemented; disable with -no-showNotImplemented
\end{itemize}

\paragraph{Options that control output}
\begin{itemize}[noitemsep,nolistsep]
\item \textbf{-quiet}: turns off all output except errors and warnings. Equivalent to \texttt{-verboseness=0}
\item \textbf{-normal}: quiet output plus a modest amount of informational and progress output. Equivalent to \texttt{-verboseneness=1}
\item \textbf{-progress}: normal output plus output about progress through the phases of activity and the files being processed. Equivalent to \texttt{-verboseneness=2}
\item \textbf{-jmlverbose}: progress output plus a verbose amount of output about the phases of activity and the files being processed. Equivalent to \texttt{-verboseneness=3}
\item \textbf{-jmldebug}: output useful only for detailed debugging (includes the jmlverbose output). Equivalent to \texttt{-verboseneness=4}
\item \textbf{-verboseness \textit{level}}: sets the verbosity level (0-4)
\item \textbf{-show}: prints out the various translated versions of the methods
\item \textbf{-verbose}: enables openJDK output
\item \textbf{-jmltesting}: adjusts the output so that test output is more stable
\end{itemize}


\section{Parsing and Type-checking}

The basic function of OpenJML is to parse and check the well-formedness of JML annotations in the context of the associated Java program.
Such checking includes conventional type-checking and checking that names are used consistently with their visibility and purity status. 

A set of Java files with JML annotations is type-checked with the command
\boxedexampleZ{java -jar \textit{\$INSTALL}/openjml.jar -check \textit{options} \textit{files}}
or
\boxedexampleZ{java -jar \textit{\$INSTALL}/openjml.jar \textit{options} \textit{files}}
since \texttt{-check} is the default action.

A key concept to understand is how class files, source files, and specification files are found and used by the OpenJML tool. This
process is described in the following subsection. The command-line options relevant to parsing and type-checking are discussed in the subsequent subsection.

\subsection{Classpaths, sourcepaths, and specification paths in OpenJML}

When a Java compiler compiles source files, it considers three types of files:
\begin{itemize}[noitemsep,nolistsep]
\item Source files listed on the command-line
\item Other source files referenced by those listed on the command-line, but not on the command-line themselves
\item Already-compiled class files
\end{itemize}
The OpenJML tool considers the same files, but also needs
\begin{itemize}[noitemsep,nolistsep]
\item Specification files associated with classes in the program
\end{itemize}

The OpenJML tool behaves in a way similar to a typical Java compiler, 
making use of three directory paths - the classpath, the sourcepath, and the specspath.  
These paths are standard lists of directories or jar files, 
separated either by colons (Unix) or semicolons (Windows).  
Java packages are subdirectories of these directories.

\begin{itemize}[noitemsep,nolistsep]
\item \texttt{classpath}: The OpenJML classpath is set using one of these alternatives, in priority order:
\begin{itemize}[noitemsep,nolistsep]
\item As the argument to the OpenJML command-line option \texttt{-classpath}
\item As the value of the Java property \texttt{org.jmlspecs.openjml.classpath}
\item As the value of the system environment variable \texttt{CLASSPATH}
\end{itemize}
\item \texttt{sourcepath}: The OpenJML sourcepath is set using one of these alternatives, in priority order:
\begin{itemize}[noitemsep,nolistsep]
\item As the argument of the OpenJML command-line option \texttt{-sourcepath}
\item As the value of the Java property \texttt{org.jmlspecs.openjml.sourcepath}
\item As the value of the OpenJML classpath (as determined above)
\end{itemize}
\item \texttt{specspath}: The OpenJML specifications path is set using one of these alternatives, in priority order:
\begin{itemize}[noitemsep,nolistsep]
\item As the argument of the OpenJML command-line option \texttt{-specspath}
\item As the value of the Java property \texttt{org.jmlspecs.openjml.specspath}
\item As the value of the OpenJML sourcepath (as determined above)
\end{itemize}
\end{itemize}

Note that with no command-line options or Java properties set, the result is simply that the system CLASSPATH is used for all of these paths. A common practice is to simply use a single directory path, specified on the command-line using \texttt{-classpath}, for all three paths.

The paths are used as follows to find relevant files:
\begin{itemize}[noitemsep,nolistsep]
\item Source files listed on the command-line are found directly in the file system. 
If the command-line element is an absolute path to a \texttt{.java} file, it is looked up in the file system as an absolute path; 
if the command-line element is a relative path, the file is found relative to the current working directory.
\item Classes that are referenced by files on the command-line or transitively by other classes in the program, can be found in one of two ways:
\begin{itemize}[noitemsep,nolistsep]
\item The source file for the class is sought as a sub-file of an element of the \texttt{sourcepath}.
\item The class file for the class is sought as a sub-file of an element of the \texttt{classpath}.
\end{itemize}
If there is both a sourcefile and a classfile present, then TBD.
\item The OpenJML tool also looks for a specification file for each source or 
class file used in the program. The specification file is a Java-like file that 
has a \texttt{.jml} suffix, but otherwise has the same name and Java package 
as the class that it specifies. The specification file used is the first .jml file 
with the correct name and package found in the sequence of directories and jar 
files that make up the specspath. 
If no such specification file is found, any specifications in 
the \texttt{.java} source file are used, if one exists 
(as found on the command-line or on the sourcepath); otherwise default specifications are used in conjunction with the class file. 
Note that if a .jml file 
is found, then any specifications in the corresponding .java file are 
(silently) ignored.
(TBD: what if the file on the command-line is not in the sourcepath?)
\end{itemize}





\subsection{Command-line options for type-checking}

\begin{itemize}[noitemsep,nolistsep]
\item \textbf{-nullableByDefault}: sets the global default to be that all declarations are implicitly \texttt{@Nullable}
\item \textbf{-nonnullByDefault}: sets the global default to be that all declarations are implicitly \texttt{@NonNull} (the default)
\item \textbf{-purityCheck}: enables (default on) checking for purity; disable with \texttt{-no-purityCheck}
\item \textbf{-internalSpecs}: enables (default on) using the built-in library specifications; disable with -\texttt{no-internalSpecs}
\item \textbf{-internalRuntime}: enables (default on) using the built-in runtime library; disable with -\texttt{no-internalRuntime}
\end{itemize}







\textit{This section will be added later.} %% TBD

\section{Static Checking and Verification}
\textit{This section will be added later.} %% TBD

\subsection{Options specific to static checking}
\begin{itemize}[noitemsep,nolistsep]
\item \textbf{-prover \textit{prover}}: the name of the prover to use: one of z3\_4\_3, yices2 [TBD: expand list]
\item \textbf{-exec \textit{path}}: the path to the executable corresponding to the given prover
\item \textbf{-boogie}: enables using boogie (-prover option ignored; -exec must specify the Z3 executable)
\item \textbf{-method \textit{methodlist}}: a comma-separated list of method names to check (default is all methods in all listed classes) [TBD - describe wildcards and fully 
\item \textbf{-exclude \textit{methodlist}}: a comma-separated list of method names to exclude from checking
\item \textbf{-checkFeasibility \textit{where}}: checks feasibility of the program at various points:
one of \texttt{none}, \texttt{all}, \texttt{exit} [TBD, finish list, give default]
\item \textbf{-escMaxWarnings \textit{int}}: the maximum number of assertion violations to look for; the argument is either a positive integer or \texttt{All} (or equivalently \texttt{all}, default is \texttt{All})
\item \textbf{-trace}: prints out a counterexample trace for each failed assert
\item \textbf{-subexpressions}: prints out a counterexample trace with model values for each subexpression
\item \textbf{-counterexample} or \textbf{-ce}: prints out counterexample information
\end{itemize}

\section{Runtime Assertion Checking}
\textit{This section will be added later.} %% TBD

\subsection{Options specific to runtime checking}
\begin{itemize}[noitemsep,nolistsep]
\item \textbf{-showNotExecutable}: warns about the use of features that are not executable (and thus ignored)
\item \textbf{-racShowSource}: includes source location in RAC warning messages [ TBD: default? ]
\item \textbf{-racCheckAssumptions}: enables (default on [TBD - is this default correct?]) checking \texttt{assume} statements as if they were asserts
\item \textbf{-racJavaChecks}: enables (default on) performing JML checking of violated Java features (which will just proceed to throw an exception anyway)
\item \textbf{-racCompileToJavaAssert}: (default off) compile RAC checks using Java asserts (which must then be enabled using \texttt{-ea}), instead of using \texttt{org.jmlspecs.utils.JmlAssertionFailure}
\item \textbf{-racPreconditionEntry}: (default off) enable distinguishing internal Precondition errors from entry Precondition errors, appropriate for automated testing; compiles code to generate JmlAssertionError exceptions
(rather than RAC warning messages)[TBD - should this turn on -racCheckAssumptions?] 
\end{itemize}

\section{Generating Documentation}
\textit{This section will be added later.} %% TBD

\section{Generating Specification File Skeletons}
\textit{This section will be added later.} %% TBD

\section{Generating Test Cases}
\textit{This section will be added later.} %% TBD

\section{Limitations of OpenJML's implementation of JML}
Currently OpenJML does not completely implement JML. The differences are explained in the following subsections.

\subsection{model import statement}
OpenJML translates a JML model import statement into a regular Java import statement [TBD - check this].
COsequently, names introduced in a model import statement are visible in both Java code and JML annotations.
This has consequences in the situation in which a name is imported both through a Java import and a JML model import.
Consider the following examples of involving packages \texttt{a} and \texttt{b}, each containing a class named 
\texttt{X}.

In these two examples,
\boxedexampleZ{
import a.X; //@ model import b.X;
}
\boxedexampleS{
import a.*; //@ model import b.*;
}
the class named \texttt{X} is imported by both an import statement and a model import statement. In JML, the use of \texttt{X} in
Java code unambiguously refers to \texttt{a.X}; the use of \texttt{X} in JML annotations is ambiguous. However, in OpenJML,
the use of \texttt{X} in both contexts will be identified as ambiguous.

In
\boxedexampleZ{
import a.*; //@ model import b.X;
}
a use of \texttt{X} in Java code refers to \texttt{a.X} and a use in JML annotations refers to \texttt{b.X}.
However, in OpenJML, both uses will mean \texttt{b.X}.

However,
\boxedexampleZ{
import a.X; //@ model import b.*;
}
is unproblematic. Both JML and OpenJML will interpret \texttt{X} as \texttt{a.X} in both Java code and JML annotations.

TBD - more to be said about .jml files

\subsection{purity checks and system library annotations}

JML requires that methods that are called within JML annotations must be pure methods (cf. section TBD). OpenJML does implement
a check for this requirement. However, to be pure, a method must be annotated as such by either \texttt{/* pure */} or 
\texttt{@Pure}. A user should insert such annotations where appropriate in the user's own code. However, many system libraries 
still lack JML annotations, including indications of purity. Using an unannotated library call within JML annotation will provoke a warning from OpenJML. Until the system libraries are more thoroughly annotated, users may wish to use the \texttt{-no-purityCheck} option to turn off purity checking.

\subsection{TBD - other unimplemented features}
