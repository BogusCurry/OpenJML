
\chapter{OpenJML tools}

\section{Parsing and Type-checking}

The foundational function of OpenJML is to parse and check the well-formedness of JML annotations in the context of the associated Java program.
Such checking includes conventional type-checking and checking that names are used consistently with their visibility and purity status. 

A set of Java files with JML annotations is type-checked with the command
\boxedexampleZ{java -jar \textit{\$INSTALL}/openjml.jar -check \textit{options} \textit{files}}
or
\boxedexampleZ{java -jar \textit{\$INSTALL}/openjml.jar \textit{options} \textit{files}}
since \texttt{-check} is the default action.
The equivalent action in the Eclipse plug-in is the 'Typecheck JML' command, available through the toolbar or menu actions.
Any \texttt{.jml} files are checked when the associated \texttt{.java} file is created. Only \texttt{.java} files either listed on the command-line or contained in folders listed on the command-line are certain to be checked. Some checking of other files may be performed where references are made to classes or methods in those non-listed files.

A key concept to understand is how class files, source files, and specification files are found and used by the OpenJML tool. This
process is described in the following subsection. The command-line options relevant to parsing and type-checking are discussed in the subsequent subsection.

\subsection{Classpaths, sourcepaths, and specification paths in OpenJML}

When a Java compiler compiles source files, it considers three types of files:
\begin{itemize}[noitemsep,nolistsep]
\item Source files listed on the command-line
\item Other source files referenced by those listed on the command-line, but not on the command-line themselves
\item Already-compiled class files
\end{itemize}
The OpenJML tool considers the same files, but also needs
\begin{itemize}[noitemsep,nolistsep]
\item Specification files associated with classes in the program
\end{itemize}

The OpenJML tool behaves in a way similar to a typical Java compiler, 
making use of three directory paths - the classpath, the sourcepath, and the specspath.  
These paths are standard lists of directories or jar files, 
separated either by colons (Unix) or semicolons (Windows).  
Java packages are subdirectories of these directories.

\begin{itemize}[noitemsep,nolistsep]
\item \texttt{classpath}: The OpenJML classpath is set using one of these alternatives, in priority order:
\begin{itemize}[noitemsep,nolistsep]
\item As the argument to the OpenJML command-line option \texttt{-classpath}
\item As the value of the Java property \texttt{org.jmlspecs.openjml.classpath}
\item As the value of the system environment variable \texttt{CLASSPATH}
\end{itemize}
\item \texttt{sourcepath}: The OpenJML sourcepath is set using one of these alternatives, in priority order:
\begin{itemize}[noitemsep,nolistsep]
\item As the argument of the OpenJML command-line option \texttt{-sourcepath}
\item As the value of the Java property \texttt{org.jmlspecs.openjml.sourcepath}
\item As the value of the OpenJML classpath (as determined above)
\end{itemize}
\item \texttt{specspath}: The OpenJML specifications path is set using one of these alternatives, in priority order:
\begin{itemize}[noitemsep,nolistsep]
\item As the argument of the OpenJML command-line option \texttt{-specspath}
\item As the value of the Java property \texttt{org.jmlspecs.openjml.specspath}
\item As the value of the OpenJML sourcepath (as determined above)
\end{itemize}
\end{itemize}

Note that with no command-line options or Java properties set, the result is simply that the system CLASSPATH is used for all of these paths. A common practice is to simply use a single directory path, specified on the command-line using \texttt{-classpath}, for all three paths.

The paths are used as follows to find relevant files:
\begin{itemize}[noitemsep,nolistsep]
\item Source files listed on the command-line are found directly in the file system. 
If the command-line element is an absolute path to a \texttt{.java} file, it is looked up in the file system as an absolute path; 
if the command-line element is a relative path, the file is found relative to the current working directory.
\item Classes that are referenced by files on the command-line or transitively by other classes in the program, can be found in one of two ways:
\begin{itemize}[noitemsep,nolistsep]
\item The source file for the class is sought as a sub-file of an element of the \texttt{sourcepath}.
\item The class file for the class is sought as a sub-file of an element of the \texttt{classpath}.
\end{itemize}
If there is both a sourcefile and a classfile present, then
\begin{itemize}[nolistsep,noitemsep]
\item if the option \texttt{-Xprefer:source} is present, the sourcefile is always recompiled
\item if the option \texttt{-Xprefer:newer} is present, the sourcefile is recompiled only if its modification timestamp is newer than that of the class file.
\end{itemize}
The default is to use the newer of the source or class files.
\end{itemize}





\subsection{Command-line options for type-checking}

The following command line options are particularly relevant to type-checking.
\begin{itemize}[noitemsep,nolistsep]
\item \textbf{-nullableByDefault}: sets the global default to be that all variable, field, method parameter, and method return type declarations are implicitly \texttt{@Nullable}
\item \textbf{-nonnullByDefault}: sets the global default to be that all variable, field, method parameter, and method return typedeclarations are implicitly \texttt{@NonNull} (the default)
\item \textbf{-purityCheck}: enables (default on) checking for purity; disable with \texttt{-no-purityCheck}
\item \textbf{-internalSpecs}: enables (default on) using the built-in library specifications; disable with -\texttt{no-internalSpecs}
\item \textbf{-internalRuntime}: enables (default on) using the built-in runtime library; disable with -\texttt{no-internalRuntime}
\end{itemize}


\section{Static Checking and Verification}
\textit{This section will be added later.} %% TBD

\subsection{Options specific to static checking}
\begin{itemize}[noitemsep,nolistsep]
\item \textbf{-prover \textit{prover}}: the name of the prover to use: one of z3\_4\_3, yices2 [TBD: expand list]
\item \textbf{-exec \textit{path}}: the path to the executable corresponding to the given prover
\item \textbf{-boogie}: enables using boogie (-prover option ignored; -exec must specify the Z3 executable)
\item \textbf{-method \textit{methodlist}}: a comma-separated list of method names to check (default is all methods in all listed classes) [TBD - describe wildcards and fully 
\item \textbf{-exclude \textit{methodlist}}: a comma-separated list of method names to exclude from checking
\item \textbf{-checkFeasibility \textit{where}}: checks feasibility of the program at various points:
one of \texttt{none}, \texttt{all}, \texttt{exit} [TBD, finish list, give default]
\item \textbf{-escMaxWarnings \textit{int}}: the maximum number of assertion violations to look for; the argument is either a positive integer or \texttt{All} (or equivalently \texttt{all}, default is \texttt{All})
\item \textbf{-trace}: prints out a counterexample trace for each failed assert
\item \textbf{-subexpressions}: prints out a counterexample trace with model values for each subexpression
\item \textbf{-counterexample} or \textbf{-ce}: prints out counterexample information
\end{itemize}

\section{Runtime Assertion Checking}
\textit{This section will be added later.} %% TBD

\subsection{Options specific to runtime checking}
\begin{itemize}[noitemsep,nolistsep]
\item \textbf{-showNotExecutable}: warns about the use of features that are not executable (and thus ignored)
\item \textbf{-racShowSource}: includes source location in RAC warning messages [ TBD: default? ]
\item \textbf{-racCheckAssumptions}: enables (default on [TBD - is this default correct?]) checking \texttt{assume} statements as if they were asserts
\item \textbf{-racJavaChecks}: enables (default on) performing JML checking of violated Java features (which will just proceed to throw an exception anyway)
\item \textbf{-racCompileToJavaAssert}: (default off) compile RAC checks using Java asserts (which must then be enabled using \texttt{-ea}), instead of using \texttt{org.jmlspecs.utils.JmlAssertionFailure}
\item \textbf{-racPreconditionEntry}: (default off) enable distinguishing internal Precondition errors from entry Precondition errors, appropriate for automated testing; compiles code to generate JmlAssertionError exceptions
(rather than RAC warning messages)[TBD - should this turn on -racCheckAssumptions?] 
\end{itemize}

\section{Generating Documentation}
\textit{This section will be added later.} %% TBD

\section{Generating Specification File Skeletons}
\textit{This section will be added later.} %% TBD

\section{Generating Test Cases}
\textit{This section will be added later.} %% TBD

\section{Limitations of OpenJML's implementation of JML}
Currently OpenJML does not completely implement JML. The differences are explained in the following subsections.

\subsection{model import statement}
OpenJML translates a JML model import statement into a regular Java import statement [TBD - check this].
COsequently, names introduced in a model import statement are visible in both Java code and JML annotations.
This has consequences in the situation in which a name is imported both through a Java import and a JML model import.
Consider the following examples of involving packages \texttt{a} and \texttt{b}, each containing a class named 
\texttt{X}.

In these two examples,
\boxedexampleZ{
import a.X; //@ model import b.X;
}
\boxedexampleS{
import a.*; //@ model import b.*;
}
the class named \texttt{X} is imported by both an import statement and a model import statement. In JML, the use of \texttt{X} in
Java code unambiguously refers to \texttt{a.X}; the use of \texttt{X} in JML annotations is ambiguous. However, in OpenJML,
the use of \texttt{X} in both contexts will be identified as ambiguous.

In
\boxedexampleZ{
import a.*; //@ model import b.X;
}
a use of \texttt{X} in Java code refers to \texttt{a.X} and a use in JML annotations refers to \texttt{b.X}.
However, in OpenJML, both uses will mean \texttt{b.X}.

However,
\boxedexampleZ{
import a.X; //@ model import b.*;
}
is unproblematic. Both JML and OpenJML will interpret \texttt{X} as \texttt{a.X} in both Java code and JML annotations.

TBD - more to be said about .jml files

\subsection{purity checks and system library annotations}

JML requires that methods that are called within JML annotations must be pure methods (cf. section TBD). OpenJML does implement
a check for this requirement. However, to be pure, a method must be annotated as such by either \texttt{/* pure */} or 
\texttt{@Pure}. A user should insert such annotations where appropriate in the user's own code. However, many system libraries 
still lack JML annotations, including indications of purity. Using an unannotated library call within JML annotation will provoke a warning from OpenJML. Until the system libraries are more thoroughly annotated, users may wish to use the \texttt{-no-purityCheck} option to turn off purity checking.

\subsection{TBD - other unimplemented features}
