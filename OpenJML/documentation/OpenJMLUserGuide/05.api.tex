
\chapter{Using OpenJML and OpenJDK within user programs}

The OpenJML software is available as a library so that Java and JML programs can be manipulated 
within a user's program. The developer needs only to include the {\tt openjml.jar} library on 
the classpath when compiling a program and to call methods through the public API as described in this chapter.
The public API is implemented in the interface {\tt org.jmlspecs.openjml.IAPI}; it provides the ability to
\begin{itemize}
\item perform compilation actions as would be executed on the command-line
\item parse files or Strings containing Java and JML source code, producing parse trees
\item print parse trees
\item walk over parse trees to perform user-defined actions
\item type-check parse trees (both Java and JML checking)
\item perform static checking
\item compile modules with run-time checks
\item emit javadoc documentation with JML annotations
\end{itemize}
The sections of this chapter describe these actions and various concepts needed to perform them correctly.

CAUTION: OpenJML relies on parts of the OpenJDK software that are labeled as internal, non-public and subject to
change. Correspondingly, some of the OpenJML API may change in the future. The definition of the API class is 
intended to provide a buffer against such changes. However, the names and functionality of OpenJDK classes (e.g.,
the {\tt Context} class in the next section) could change.

\paragraph{List classes}
CAUTION \#2: The OpenJDK software uses its own implementation of Lists, namely {\tt com.sun.tools.javac.util.List}. 
It is a different implementation than {\tt java.util.List}, with a different interface. Since one or the other may
 be in the list of imports, the use of {\tt List} in the code may not clearly indicate which type of List is being
 used. Error messages are not always helpful here. Users should keep these two types of List in mind to avoid 
 confusion.

\paragraph{Example source code}
The subsections that follow contain many source code examples. 
Small source code snippets are shown in in-line boxes like this:
\boxedexample{ // A Java comment }
Larger examples are shown as full programs. These are followed by a box of text with a gray background that contains the output expected if the program is run (if the program is error-free) or compiled (if there are compilation errors).
Here is a ``Hello, world'' example program:
\demo{DemoHelloWorld}

\noindent
All of these full-program example programs are working, tested examples.
They are available in the {\tt demos} directory of the OpenJML source code. The opening comment
line (as well as the class name) of the example text gives the file name.

The full programs presume an appropriate environment. In particular, they expect the following
\begin{itemize}
\item the current working directory is the {\tt demos} directory of the OpenJML source distribution
\item the Java {\tt CLASSPATH} contains the current directory and a release version of the
OpenJML library ({\tt openjml.jar}). For example, if the demos directory is the current working directory and a copy of {\tt openjml.jar} is in the 
{\tt demos} directory, then the {\tt CLASSPATH} could be set as {\tt ``.;openjml.jar''} (using the ; on Windows, a : on Mac and Linux)
\end{itemize}
Note that the examples often use other files that are in subdirectories of the {\tt demos} directory.
\boxedexampleZ{ % FIXME - this does not translate well to HTML
// bash commands to compile and run the DemoHelloWorld example
\begin{tabular}{ll}
cd OpenJML/demos     &\# Alter this to match your local installation \\
export CLASSPATH=''.;openjml.jar''     &\# Use a : instead of ; on Unix or Mac \\
                                       &\# Copy openjml.jar to the demo directory \\
javac DemoHelloWorld.java              &\# Be sure java tools from a 1.7 JDK  \\
java DemoHelloWorld                    &\# are on the PATH
\end{tabular}
}

\section{Concepts}

\subsection{Compilation Contexts}
All parsing and compilation activities within OpenJML are performed with respect to a {\em compilation context},
implemented in the code as a {\tt com.sun.tools.javac.util.Context} object. There can be more than one Context at a given time,
though this is rare. A context holds all of the symbol tables and cached values that represent the source code created
in that context. 

There is little need for the user to create or manipulate
Contexts. However it is essential that items created in one Context not be used in another context. There is no check for such misuse, but the subsequent actions are likely to fail. For example, a Context contains interned versions of the names of
source code identifiers (as {\tt Name}s). Consequently an identifier parsed in one Context will 
appear different than an identifier
parsed in another Context, even if they have the same textual name. Do not try to reuse parse trees or other objects 
created in one Context in another Context.
 
Each instance of the {\tt IAPI} interface creates its own Context object and most methods on 
that {\tt IAPI} instance operate with respect to that Context. The {\tt API.close} operation 
releases the Context object, allowing the garbage collector to reclaim space.
\footnote{The OpenJDK software was designed as a command-line tool, in which all memory is reclaimed
when the process exits. Although in principle memory can be garbage collected when no more references
to a Context or its consitutent parts exist, the degree to which this is the case has not been tested.}

\label{JavaFileObjects}
\subsection{JavaFileObjects}

OpenJDK works with source files using {\tt JavaFileObject} objects. This class abstracts the behavior of 
ordinary source files. Recall that the definition of the Java language allows source material to be held 
in containers other than ordinary files on disk; The {\tt JavaFileObject} class accommodates such implementations.

OpenJML currently handles source material in ordinary files and source material expressed as {\tt String} objects
and contained in mock-file objects. Such mock objects make it easier to create source material programatically, 
without having to create temporary files on disk.

Although the basic input unit to OpenJDK and OpenJML is a JavaFileObject, for convenience, methods that require
source material as input have variations allowing the inputs to be expressed as names of files or {\tt File} objects.
If needed, the following methods create JavaFileObjects:
\boxedexampleA{
String filename = ... \\
File file = new java.io.File(filename);\\
IAPI m = Factory.makeAPI();\\
JavaFileObject jfo1 = m.makeJFOfromFilename(filename);\\
JavaFileObject jfo2 = m.makeJFOfromFile(file);\\
JavaFileObject jfo3 = m.makeJFOfromString(filename,contents);
}
The last of the methods above, {\tt makeJFOfromString}, creates a mock-file object with the given contents (a String).
The {\tt contents} argument is a String holding the text that would be in a compilation unit.
The mock-object must have a sensible filename as well. In particular, the given filename should match the package and
class name as given in the {\tt contents} argument. In addition to creating the {\tt JavaFileObject} object, the 
mock-file is also added to an internal database of source mock-files; if a mock-file has a filename that would be on
the source path (were it a concrete file), then the mock-file is used as if it were a real file in an OpenJML compilation.
[TODO: Test this. Also, how to remove such files from the internal database. ] 

\subsection{Interfaces and concrete classes}

A design meant to be extended should preferably be expressed as Java interfaces; if client code uses the interface and not the underlying concrete classes, then reimplementing functionality with new classes is straightforward. The OpenJDK 
architecture uses interfaces in some places, but often it is the concrete classes that must be extended.

Table \ref{Interfaces} lists important interfaces, the corresponding OpenJDK concrete class, and the OpenJML replacement.

\begin{table}
\begin{tabular}{|l|l|l|}
\hline
Interface & OpenJDK class & OpenJML class \\ \hline
IAPI &  & API \\ \hline
 & com.sun.tools.javac.main.Main & org.jmlspecs.openjml.Main \\ \hline
 & Option & \\ \hline
IOption & & JmlOption \\ \hline
IVisitor & & \\ \hline
IJmlTree & & \\ \hline
IJmlVisitor & & \\ \hline
IProver & & \\ \hline
IProverResult & & ProverResult \\ \hline
IProverResult.ICounterexample & & Counterexample \\ \hline
IProverResult.ICoreIds & & \\ \hline
JCDiagnostic.DiagnosticPosition & SimpleDiagnosticPosition & DiagnosticPositionSE, DiagnosticPositionSES \\ \hline
Diagnostic<T> & JCDiagnostic & \\ \hline
 & com.sun.tools.javac.main.JavaCompiler & JmlCompiler \\ \hline
 & & \\ \hline
 & & \\ \hline
\end{tabular}
\caption{Interfaces and Classes}
\label{Interfaces}
\end{table}

TODO: Add Parser, Scanner, other tools, JCTree nodes, JMLTree nodes, Option/JmlOption, DiagnosticPosition, Tool, OptionCHecker


\subsection{Object Factories}

\subsection{Abstract Syntax Trees}

\subsection{Compilation Phases and The tool registry}

Compilation in the OpenJDK compiler proceeds in a number of phases. Each phase is implemented by a specific tool.
OpenJDK examples are the {\tt DocCommentScanner}, {\tt EndPosParser}, {\tt Flow}, performing scanning, parsing and flow checks respectively; the OpenJML counterparts are {\tt JmlScanner}, {\tt JmlParser}, and {\tt JmlFlow}.

In each compilation context there is one instance of each tool, registered with the context. The Context contains a map
of keys to the singleton instance of the tool (or its factory) for that context. The scanner and parser are treated slightly differently: there is a singleton instance of a scanner factory and a parser factory, but a new instance of the
scanner and the parser are created for each compilation unit compiled. Tables \ref{Tools1} and \ref{Tools2} list the tools
most likely to be encounterded when programming with OpenJML.

OpenJML implements
alternate versions of many of the OpenJDK tools. The OpenJML versions are derived from the OpenJDK versions and are 
registered in the context in place of the OpenJDK versions. In that way, anywhere in the software that a tool is
obtained (using the syntax {\tt ZZZ.instance(context)} for a tool {\tt ZZZ}), the appropriate version and instance
of the tool
is produced.

In some cases, a {\em tool factory} is registered instead of a tool instance. Then a tool instance is created on the
first request for an instance of the tool. The reason for this is the following. Most tools use other tools and, for
efficiency, request instances of those tools in their constructors. Circular dependencies can easily arise among these
tool dependencies. Using factories helps mitigate this, though the problem still does easily arise.



\begin{table}
\begin{tabularx}{\textwidth}{|p{1.5in}|p{1.5in}|X|}

\hline
Purpose & Java and JML tool &  Notes \\
\hline
overall compiler & JavaCompiler,\newline JmlCompiler & controls the flow of compilation phases \\
\hline
scanner factory & ScannerFactory,\newline JmlScanner.Factory  & \\
\hline
Token scanning & DocCommentScanner,\newline JmlScanner &  new instance created from the factory for each compilation unit \\
\hline
parser factory & ParserFactory,\newline JmlFactory  & \\
\hline
parser & EndPosParser,\newline JmlParser & new instance created from the factory for each compilation unit \\
\hline
symbol table construction & Enter,\newline JmlEnter & \\
\hline
annotation processing & Annotate & performed in JavaCompiler.processAnnotations\\
\hline
type determination and checking & Attr,\newline JmlAttr & \\
\hline
flow-sensitive checks & Flow,\newline JmlFlow & simple type-checking stops here \\
\hline
\hline
static checking & \hspace{.1in} \newline JmlEsc & invoked instead of desugaring if static checking is enabled (and processing ends here) \\
\hline
\hline
runtime assertion checking & \hspace*{.1in}\newline JmlRac & invoked if RAC is enabled, and then proceeds with the remainder of compilation and code generation \\
\hline
desugaring generics & & performed in the method JavaCompiler.desugar \\
\hline
code generation & Gen & not used for ESC \\
\hline


\end{tabularx}
\caption{Compilation phases and corresponding tools as implemented in JavaCompiler and JmlCompiler}
\label{Tools1}
\end{table}

\begin{table}[bth]
\begin{center}
\begin{tabular}{|l|l|l|}

\hline
Purpose & Java and JML tool &  Notes \\
\hline
identifier table & Names & \\
\hline
symbol table & SymTab & \\
\hline
compiler and command-line options & Options, JmlOptions & \\
\hline
AST node factory & JCTree.Factory, JmlTree.Maker & \\
\hline
message reporting & Log & \\
\hline
printing ASTs & Pretty, JmlPretty & \\
\hline
name resolution & Resolve, JmlResolver & \\
\hline
AST utilities & TreeInfo, JmlTreeInfo & \\
\hline
type checks & Check, JmlCheck & \\
\hline
creating diagnostic message objects & JCDiagnostic.Factory & \\
\hline
\end{tabular}
\end{center}
\caption{Some of the other registered tools}
\label{Tools2}
\end{table}

TBD: Others - MemberEnter, JmlMemberEnter, JmlRac, JmlCheck, Infer, Types, Options, Lint, Source, JavacMessages, DiagnosticListener, JavaFileManager/JavacFileManager, ClassReader/javadocClassReader, JavadocEnter, DocEnv/DocEnvJml, BasicBlocker, ProgressReporter?, ClassReader, ClassWriter, Todo, Annotate, Types, TaskListener, JavacTaskImpl, JavacTrees

TBD: Others - JmlSpecs, Utils, Nowarns, JmlTranslator, Dependencies

TBD: Is JmlTreeInfo still used

\section{OpenJML operations}

\subsection{Methods equivalent to command-line operations}

The {\tt execute} methods of {\tt IAPI} perform the same operation as a command on the command-line.
These methods are different than others of {\tt IAPI} in that they create and use their own {\tt Context}
object, ignoring that of the calling {\tt IAPI} object.

The simple method is shown here:
\boxedexampleA{
import org.jmlspecs.openjml.IAPI;\\
\\
IAPI m = new org.jmlspecs.openjml.API(); \\
int returnCode = m.execute(``-check'',''-noPurityCheck'',''src/demo/Err.java'');
}
\noindent
Each argument that would appear on the command-line is a separate argument to {\tt execute}.
All informational and diagnostic output is sent to {\tt System.out}.
The value returned by {\tt execute} is the same as the exit code returned by the equivalent command-line operation.
The String arguments are a varargs list, so they can be provided to {\tt execute} as a single array:
\boxedexampleA{
import org.jmlspecs.openjml.IAPI; \\
String[] args = new String[]\{``-check'',''-noPurityCheck'',''src/demo/Err.java''\}; \\
IAPI m = new org.jmlspecs.openjml.API(); \\
int returnCode = m.execute(args);
}
\noindent
A full example of using execute on a file with a syntax error is shown below:
\boxedinputc{\source/DemoExecute.java}{\source/DemoExecute.txt}

\noindent
A longer form of {\tt execute} takes two additional arguments: a {\tt Writer} and a
{\tt DiagnosticListener}. The {\tt Writer} receives all the informational output.
The {\tt report} method of the {\tt DiagnosticListener} is called for each warning or
error diagnostic generated by OpenJML. Here is a full example of this method:
\demo{DemoExecute2}


\subsection{Parsing}

There are two varieties of parsing. The first parses an individual Java or specification file, producing an AST that
represents that source file. The second parses both a Java file and its specification file, if there is a separate one.
The second form is generally more useful, since the specification file is found automatically. However, if the parse trees
are being constructed programmatically, it may be useful to parse the files individually and then manually associate them.

Parsing constructs a parse tree. No symbols are created or entered into a symbol table. Nor is any type-checking performed.
The only global effect is that identifiers are interned in the {\tt Names} table, which is specific to the compilation context.
Thus the only effect of discarding a parse tree is that there may be orphaned (no longer used) names in the {\tt Names} table.
The {\tt Names} table cannot be cleared without the risk of dangling identifiers in parse trees.

Other than this consideration, parse trees can be created, manipulated, edited and discarded. Section TBD describes tools for
manually creating parse trees and walking over them. Once a parse tree is type-checked, it should be considered immutable.

\subsubsection{Parsing individual files}

There are two methods for parsing an individual file. The basic 
method takes a {\tt JavaFileObject} as input and produces an AST.
The convenience method takes a filename as input and produces an AST. The methods of section \ref{JavaFileObjects} enable you to
produce {\tt JavaFileObject}s from filenames, File objects, or Strings that hold the equivalent of the contents of a file (a compilation unit).
\boxedexampleA{
JmlCompilationUnit parseSingleFile(String filename);\\
JmlCompilationUnit parseSingleFile(JavaFileObject jfo);
}
The filename is relative to the current working directory.
%% TBD: Change to reflect interfaces; check the JFO signature

Here is a full example that shows both interfaces and shows how to attach a specification parse tree to its Java parse tree.
\demo{DemoParseSingle}


%% TBD: How to attach specs to binary class files?

\subsubsection{Parsing Java and JML files together}

The more common action is to parse a Java file and its specification at the same time. The JML language defines how the specification file is found for a given source or binary class. In short, the specification file has syntax very similar to 
a Java file:
\begin{itemize}
\item it must be in the same package and have the same class name as the Java class
\item if both are files, the filenames without suffix must be the same
\item the specification file must be on the {\em specspath}
\item if a .jml file meeting the above criteria is found anywhere on the specspath, it is used; otherwise a .java file on the specspath meeting
the above criteria is used; otherwise only default specifications are used.\footnote{In the past, JML allowed multiple specification files and defined an ordering and rules for combining the specifications contained in them. The JML has been 
simplified to allow just one specification file, just one suffix (.jml), and no combining of specifications from a .jml and a .java file if both exist.}
\end{itemize}
Note that a Java file can be specified on the command-line that is not on the specspath. In that case (if there is no .jml file) 
no specification file will be found, although the user may expect that the Java file itself may serve as its own specifications.
This is a confusing situation and should be avoided.

%% TBD - more needed

\subsection{Type-checking}

\subsection{Static checking}

\subsection{Compiling run-time checks}

\subsection{Creating JML-enhanced documentation}

\section{Working with ASTs}

\subsection{Printing parse trees}

TBD

\subsection{Source location information}

TBD

\subsection{Exploring parse trees with Visitors}
OpenJML defines some Visitor classes that can be extended to implement user-defined functionality while traversing a parse tree.
The basic class is {\tt JmlScanner}. An unmodified instance of {\tt JmlScanner} will traverse a parse tree without performing
any actions.

There are three modes of traversing an AST.
\begin{itemize}
\item AST\_JAVA\_MODE - traverses only the Java portion of an AST, ignoring any JML annotations
\item AST\_JML\_MODE - traverses the Java and JML syntax that was part of the original source file
\item AST\_SPEC\_MODE - traverses the Java syntax and its specifications (whether they came from the same source file or a different one). This mode is only available after the AST has been type-checked.
\end{itemize}

A derived class can affect the behavior of the visitor in two ways:
\begin{itemize}
\item By overriding the {\tt scan} method, an action can be performed at every node of an AST
\item By overriding specific {\tt visit...} methods, an action can be performed that is specific to the nodes of the corresponding type
\end{itemize}

In the example that follows, the scan method of the Visitor is modified to print the node type and count all nodes in the AST, the visitBinary method
is modified to count Java binary operations, and the visitJmlBinary method is modified to count JML binary operations.
The default constructor of the parent Visitor class sets the traversal mode to AST\_JML\_MODE.
\demo{DemoWalkTree1}

The second example shows the differences among the three traversal modes. Note that the AST\_SPEC\_MODE traversal fails when requested prior to type-checking the AST.
\demo{DemoWalkTree2}

There are two other points to make about these examples.
\begin{itemize}
\item Note that each derived method calls the superclass version of the method that it overrides. The superclass method implements the logic to traverse all the children of the AST node. If the super call is omitted, no traversal of the children is performed. If the derived class wishes to traverse only some of the children, a specialized implementation of the method will need to be created.
It is easiest to create such an implementation by consulting the code in the super class.
\item In the examples above, you can see that the System.out.println statement that prints the node's class occurs before the super call. The result is a pre-order traversal of the tree; if the print statement occurred after the super call, the output would show a post-order traversal.
\end{itemize}

%% TBD - comment on other visitor classes

\subsection{Creating parse trees}

\section{Working with JML specifications}

\section{Utilities}

-- version
-- context
-- symbols

\section{Extending or modifying JML}
JML is modified by providing new implementations of key classes, typically by derivation from those that are part of OpenJML.
In fact, OpenJML extends many of the OpenJDK classes to incorporate JML functionality into the OpenJDK Java compiler.

\subsection{Adding new command-line options}
\subsection{Altering IAPI}
\subsection{Changing the Scanner}
\subsection{Enhancing the parser}
\subsection{Adding new modifiers and annotations}
\subsection{Adding new AST nodes}
\subsection{Modifying a compiler phase}


